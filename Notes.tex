\documentclass[oneside, 11pt]{book}
\usepackage{fancyhdr, enumitem, amsfonts, amsmath, array, graphicx, setspace, xcolor}
\pagestyle{fancyplain}
\fancyfoot[C]{\thepage}
\setlength{\topmargin}{0mm}
\setlength{\oddsidemargin}{0mm}
\setlength{\evensidemargin}{0mm}
\setlength{\textheight}{225mm}
\setlength{\textwidth}{165mm}
\setlength{\marginparwidth}{0mm}
\setlength{\marginparsep}{0mm}
\setlength{\marginparpush}{0mm}
\setlength{\parindent}{0mm}
\setlength{\footskip}{0mm}
\setlength{\headwidth}{165mm}

\newcommand{\indep}{\perp \!\!\! \perp}
\newtheorem{dfn}{Definition}[section]

\renewcommand{\headrulewidth}{0.15mm}
\renewcommand{\footrulewidth}{0.15mm}
\renewcommand\thesection{\arabic{section}}
\renewcommand{\chaptermark}[1]{\markboth{\thechapter. #1}{}}
\renewcommand{\sectionmark}[1]{\markright{\thesection. #1}}
\lhead[]{}
\rhead[]{}

\begin{document}
	% Cover page
	\thispagestyle{empty} 
	\centering \hspace{-1mm} \vfill {\Huge\bf Theory of Algorithms} 
	\vfill
	% End of cover page
	\newpage
	% ToC
	\doublespacing
	\setcounter{tocdepth}{3}
	\tableofcontents
	\thispagestyle{empty}
	% End of ToC
	\newpage
	\setcounter{page}{1}
	% Start of Page 1
	\begin{flushleft}
		\subsubsection{Divide and Conquer}
		This approach recursively breaks down a problem into two or more sub-problems of the same or related type, until these become simple enough to be solved directly. The solutions to the sub-problems are then combined to give a solution to the original problem. The steps in applying this technique:
		\begin{itemize}
			\item[1.] Divide: split the input apart into multiple smaller pieces, recursively solving each piece
			\item[2.] Conquer: combine the solutions to each smaller piece together into the overall solution
		\end{itemize}
		Problems to solve:
		\begin{itemize}
			\item merge sort
			\item quicksort
			\item sum of numbers in array
			\item closest pair (split and solve, straddle zone, quick hull)
		\end{itemize}
	\end{flushleft}
\end{document}
